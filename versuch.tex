
\documentclass[a4paper]{article}
\usepackage[utf8]{inputenc}
\usepackage[ngerman]{babel}

\begin{document}
\author{Sebastian Bernhard}
\title{Handlungsfäden in der Wirtschaftsgeschichte aus heutiger Perspektive}
\maketitle
\tableofcontents

\section{Wirtschaft als soziale Praktik}

Der Begriff sozialer Praktik, wie ihn Andreas Reckwitz in
\cite{reckwitz2003praktiken} verwendet, kann auch problemlos
alle ökonomischen Handlungen mit einbeziehen. Damit wird
wird das Verständnis von Wirtschaft auf eine zeitgemäße
Grundlage gestellt.

\input{org.tex}

\bibliographystyle{alpha}
\bibliography{literatur}
 
\end{document}
