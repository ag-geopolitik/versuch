
\documentclass[a4paper]{book}%hitec}
\usepackage[utf8]{inputenc}
\usepackage{lmodern}
\usepackage{ae}
\usepackage[german]{babel}

\usepackage{hyperref}
\usepackage{mdwlist}
\usepackage{quotmark}
% \tqt{..} oder \begin{qt} ... \end{qt}
% \usepackage{quotchap}
\usepackage{glossaries}

\providecommand\chapter[1]{\section{#1}}
%\newcommand\part[1]{\section{#1}}
% das ist wichtig!
\usepackage[utf8]{inputenc}

\hyphenation{%
  be-triebs-wirt-schaft-lich-en%
  Lebens-um-stände%
}


\makeglossaries

\begin{document}

\author{Sebastian Bernhard}
\title{Handlungsfäden in der Wirtschaftsgeschichte aus heutiger Perspektive}
\maketitle
\tableofcontents

\part{Verschiedene Blickwinkel}

\input{praktik.tex}

\input{institutionen.tex}

\input{staaten.tex}

\input{konzept.tex}

\input{entwicklung.tex}

\input{eigentum.tex}

\input{heute.tex}

\input{arbeitundleben.tex}

%\input{org.tex}

\input{wirtschaftslehre.tex}

\input{industriegesellschaft.tex}

\input{fragen.tex}

\input{methoden.tex}

\input{kapitalismus.tex}

\bibliographystyle{alpha}
\bibliography{literatur}
 
\printglossaries{}

\end{document}
